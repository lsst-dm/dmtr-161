\documentclass[DM,lsstdraft,STR,toc]{lsstdoc}
\usepackage{geometry}
\usepackage{longtable,booktabs}
\usepackage{enumitem}
\usepackage{arydshln}

\input meta.tex

\providecommand{\tightlist}{
  \setlength{\itemsep}{0pt}\setlength{\parskip}{0pt}}

\begin{document}

\def\milestoneName{LSP with Authentication and TAP}
\def\milestoneId{LDM-503-10a:}
\def\product{LSP Services}

\setDocCompact{true}

\title{ LDM-503-10a: LSP with Authentication and TAP Test Plan and Report}
\setDocRef{\lsstDocType-\lsstDocNum}
\date{\vcsdate}
\setDocUpstreamLocation{\url{https://github.com/lsst/lsst-texmf/examples}}
\author{ Gregory Dubois-Felsmann }

\input history_and_info.tex


\setDocAbstract{
This is the test plan and report for LDM-503-10a: (LSP with Authentication and TAP), an LSST level 2 milestone pertaining to the Data Management Subsystem.
}


\maketitle

\section{Introduction}
\label{sect:intro}


\subsection{Objectives}
\label{sect:objectives}

Verify the integration of federated authentication and authorization
into the LSST science platform, and the availability of an IVOA TAP
service.\\[2\baselineskip]\textbf{Milestone
Description}\\[2\baselineskip]This test demonstrates the successful
integration of a single-sign-on federated authentication system, and a
basic authorization system, with the three Aspects of the LSST Science
Platform (Portal, Notebook, and API), with the API Aspect containing at
least a TAP service. ~It will be demonstrated on a Kubernetes cluster
provided by NCSA. ~It is not required for authorization to be applied at
the database level; it is sufficient for this milestone for it to apply
at the TAP level. ~Data served will remain that from the original PDAC
work, i.e., SDSS Stripe 82 and/or WISE.\\[2\baselineskip]

\subsection{Scope}\label{scope}

The overall strategy for testing and verification within LSST Data
Management is described in \citeds{LDM-503}.\\
This test plan specifically verifies successful completion of milestone
LDM-503-10a.



\subsection{System Overview}
\label{sect:systemoverview}

The LSST Science Platform (see \citeds{LSE-319}, \citeds{LDM-554}, and \citeds{LDM-542}) is the
means of access for science users to the LSST data. ~It also serves
project-internal users for a wide variety of data access needs during
construction (using prototypes and early versions), commissioning, and
operations.


\subsection{Document Overview}
\label{sect:docoverview}

This document was generated from Jira, obtaining the relevant information from the 
\href{https://jira.lsstcorp.org/secure/Tests.jspa#/testPlan/LVV-P48}{LVV-P48}
~Jira Test Plan and related Test Cycles (
  \href{https://jira.lsstcorp.org/secure/Tests.jspa#/testCycle/LVV-C85}{LVV-C85}
).

Section \ref{sect:intro} provides an overview of the test campaign, the system under test (\product{}), the applicable documentation, and explains how this document is organized.
Section \ref{sect:configuration}  describes the configuration used for this test.
Section \ref{sect:personnel} describes the necessary roles and lists the individuals assigned to them.
%Section \ref{sect:plannedtestactivities} provides the list of planned test cycles and test cases, including all relevant information that fully describes the test campaign.

Section \ref{sect:overview} provides a summary of the test results, including an overview in Table \ref{table:summary}, an overall assessment statement and suggestions for possible improvements.
Section \ref{sect:detailedtestresults} provides detailed results for each step in each test case.

The current status of test plan LVV-P48 in Jira is \textbf{ Draft }.

\subsection{References}
\label{sect:references}
\renewcommand{\refname}{}
\bibliography{lsst,refs,books,refs_ads}
\section{Test Configuration}
\label{sect:configuration}

\subsection{Data Collection}

  Observing is not required for this test campaign.

\subsection{Verification Environment}
\label{sect:hwconf}
  The ``lsst-lsp-stable'' instance of the LSP, hosted at the LDF.





\newpage
\section{Personnel}
\label{sect:personnel}

The following personnel are involved in this test activity:

\begin{itemize}
\item Test Plan (LVV-P48) owner: Gregory Dubois-Felsmann
\item Test Cycles:
\begin{itemize}
  \item LVV-C85 owner: 
    Gregory Dubois-Felsmann
  \begin{itemize}
    \item Test case \href{https://jira.lsstcorp.org/secure/Tests.jspa#/testCase/LVV-T1437}{LVV-T1437} tester: 
    \item Test case \href{https://jira.lsstcorp.org/secure/Tests.jspa#/testCase/LVV-T1436}{LVV-T1436} tester: 
    \item Test case \href{https://jira.lsstcorp.org/secure/Tests.jspa#/testCase/LVV-T1334}{LVV-T1334} tester: 
  \end{itemize}
\end{itemize}
\item Additional Test Personnel involved:
  \begin{itemize}
    \item Test case \href{https://jira.lsstcorp.org/secure/Tests.jspa#/testCase/LVV-T1334}{LVV-T1334}: 
    \item Test case \href{https://jira.lsstcorp.org/secure/Tests.jspa#/testCase/LVV-T1436}{LVV-T1436}: 
    \item Test case \href{https://jira.lsstcorp.org/secure/Tests.jspa#/testCase/LVV-T1437}{LVV-T1437}: 
  \end{itemize}
\end{itemize}

\newpage

\section{Overview of the Test Results}
\label{sect:overview}

\subsection{Summary}
\label{sect:summarytable}

\begin{longtable}{p{0.12\textwidth}p{0.2\textwidth}p{0.56\textwidth}p{0.12\textwidth}}
\toprule

  \multicolumn{3}{c}{ Test Cycle {\bf LVV-C85: LDM-503-10a: LSP with Authentication and TAP
 }} \\\hline

  {\bf \footnotesize test case} & {\bf \footnotesize status} & {\bf \footnotesize comment} & {\bf \footnotesize issues} \\\toprule

    \href{https://jira.lsstcorp.org/secure/Tests.jspa#/testCase/LVV-T1437}{LVV-T1437}
    & Not Executed &  &
    \\\hline
    \href{https://jira.lsstcorp.org/secure/Tests.jspa#/testCase/LVV-T1436}{LVV-T1436}
    & Not Executed &  &
    \\\hline
    \href{https://jira.lsstcorp.org/secure/Tests.jspa#/testCase/LVV-T1334}{LVV-T1334}
    & Not Executed &  &
    \\\hline

\caption{Test Results Summary}
\label{table:summary}
\end{longtable}

\subsection{Overall Assessment}
\label{sect:overallassessment}

Not yet available.

\subsection{Recommended Improvements}
\label{sect:recommendations}

Not yet available.

\newpage
\section{Detailed Test Results}
\label{sect:detailedtestresults}


  \subsection{Test Cycle LVV-C85 }

Open test cycle {\it \href{https://jira.lsstcorp.org/secure/Tests.jspa#/testrun/LVV-C85}{LDM-503-10a: LSP with Authentication and TAP
}} in Jira.

  LDM-503-10a: LSP with Authentication and TAP
\\
  Status: Not Executed

  Execute the test cases associated with the DM milestone LDM-503-10a.


  \subsubsection{Software Version/Baseline}
    Not provided.

  \subsubsection{Configuration}
    Not provided.

  \subsubsection{Test Cases in LVV-C85 Test Cycle}


    \paragraph{Test Case LVV-T1437 - LDM-503-10a: API Aspect tests for LSP with Authentication and TAP
milestone
 }\mbox{}\\

Open  \href{https://jira.lsstcorp.org/secure/Tests.jspa#/testCase/LVV-T1437}{\textit{ LVV-T1437 } }
test case in Jira.

    This test case verifies that the TAP service in the API Aspect of the
Science Platform is accessible to authorized users through a login
process, and that TAP searches can be performed using the IVOA TAP
protocol from remote sites.


    \textbf{ Preconditions}:\\
    

    Execution status: {\bf Not Executed }

    Final comment:\\


    Detailed step results:

    \begin{longtable}{p{1cm}p{2cm}p{13cm}}
    \hline
    {Step} & \multicolumn{2}{c}{Description, Results and Status}\\ \hline
      1 & Description &

      \begin{minipage}[t]{13cm}{\footnotesize
      On the local computer, clone the TBD test notebook for LDM-503-10a into
the user environment from the TBD tag of the TBD Github repository.
~Note the SHA that applies to the version of the test notebook that has
been cloned.

      \vspace{\dp0}
      } \end{minipage} \\
      \\ \cdashline{2-3}


      & Expected Result &

      \begin{minipage}[t]{13cm}{\footnotesize
      
      \vspace{\dp0}
      } \end{minipage} \\
      \\ \cdashline{2-3}

      & \begin{minipage}[t]{2cm}{Actual\\ Result}\end{minipage}   & 
      \begin{minipage}[t]{13cm}{\footnotesize
      
      \vspace{\dp0}
      } \end{minipage} \\
      \\ \cdashline{2-3}


      & Status          & Not Executed \\ \hline

      2 & Description &

      \begin{minipage}[t]{13cm}{\footnotesize
      Launch a LOCAL instance of JupyterLab (i.e., by running ``jupyter lab'')
on the computer to be used for testing. ~Ensure that the test notebook
is visible from within JupyterLab.

      \vspace{\dp0}
      } \end{minipage} \\
      \\ \cdashline{2-3}


      & Expected Result &

      \begin{minipage}[t]{13cm}{\footnotesize
      
      \vspace{\dp0}
      } \end{minipage} \\
      \\ \cdashline{2-3}

      & \begin{minipage}[t]{2cm}{Actual\\ Result}\end{minipage}   & 
      \begin{minipage}[t]{13cm}{\footnotesize
      
      \vspace{\dp0}
      } \end{minipage} \\
      \\ \cdashline{2-3}


      & Status          & Not Executed \\ \hline

      3 & Description &

      \begin{minipage}[t]{13cm}{\footnotesize
      Obtain an access token for the TAP service from the LSP instance under
test, by navigating to the XXX endpoint in a web browser and logging in.
~NCSA credentials for the tester should be used.

      \vspace{\dp0}
      } \end{minipage} \\
      \\ \cdashline{2-3}


      & Expected Result &

      \begin{minipage}[t]{13cm}{\footnotesize
      
      \vspace{\dp0}
      } \end{minipage} \\
      \\ \cdashline{2-3}

      & \begin{minipage}[t]{2cm}{Actual\\ Result}\end{minipage}   & 
      \begin{minipage}[t]{13cm}{\footnotesize
      
      \vspace{\dp0}
      } \end{minipage} \\
      \\ \cdashline{2-3}


      & Status          & Not Executed \\ \hline

      4 & Description &

      \begin{minipage}[t]{13cm}{\footnotesize
      Open the test notebook and paste the access token into the appropriate
cell in the notebook. ~The text ``LVV-T1437'' should be found in the
notebook just before the appropriate line of code.

      \vspace{\dp0}
      } \end{minipage} \\
      \\ \cdashline{2-3}


      & Expected Result &

      \begin{minipage}[t]{13cm}{\footnotesize
      
      \vspace{\dp0}
      } \end{minipage} \\
      \\ \cdashline{2-3}

      & \begin{minipage}[t]{2cm}{Actual\\ Result}\end{minipage}   & 
      \begin{minipage}[t]{13cm}{\footnotesize
      
      \vspace{\dp0}
      } \end{minipage} \\
      \\ \cdashline{2-3}


      & Status          & Not Executed \\ \hline

      5 & Description &

      \begin{minipage}[t]{13cm}{\footnotesize
      Execute the entire notebook.

      \vspace{\dp0}
      } \end{minipage} \\
      \\ \cdashline{2-3}


      & Expected Result &

      \begin{minipage}[t]{13cm}{\footnotesize
      
      \vspace{\dp0}
      } \end{minipage} \\
      \\ \cdashline{2-3}

      & \begin{minipage}[t]{2cm}{Actual\\ Result}\end{minipage}   & 
      \begin{minipage}[t]{13cm}{\footnotesize
      
      \vspace{\dp0}
      } \end{minipage} \\
      \\ \cdashline{2-3}


      & Status          & Not Executed \\ \hline

      6 & Description &

      \begin{minipage}[t]{13cm}{\footnotesize
      Note the success and/or failure indications that appear in the output of
the notebook.

      \vspace{\dp0}
      } \end{minipage} \\
      \\ \cdashline{2-3}


      & Expected Result &

      \begin{minipage}[t]{13cm}{\footnotesize
      
      \vspace{\dp0}
      } \end{minipage} \\
      \\ \cdashline{2-3}

      & \begin{minipage}[t]{2cm}{Actual\\ Result}\end{minipage}   & 
      \begin{minipage}[t]{13cm}{\footnotesize
      
      \vspace{\dp0}
      } \end{minipage} \\
      \\ \cdashline{2-3}


      & Status          & Not Executed \\ \hline

      7 & Description &

      \begin{minipage}[t]{13cm}{\footnotesize
      Delete the access token from the test notebook.

      \vspace{\dp0}
      } \end{minipage} \\
      \\ \cdashline{2-3}


      & Expected Result &

      \begin{minipage}[t]{13cm}{\footnotesize
      
      \vspace{\dp0}
      } \end{minipage} \\
      \\ \cdashline{2-3}

      & \begin{minipage}[t]{2cm}{Actual\\ Result}\end{minipage}   & 
      \begin{minipage}[t]{13cm}{\footnotesize
      
      \vspace{\dp0}
      } \end{minipage} \\
      \\ \cdashline{2-3}


      & Status          & Not Executed \\ \hline

      8 & Description &

      \begin{minipage}[t]{13cm}{\footnotesize
      Save and close the test notebook. ~Save the fully-executed notebook in
TBD location as a record of the test.

      \vspace{\dp0}
      } \end{minipage} \\
      \\ \cdashline{2-3}


      & Expected Result &

      \begin{minipage}[t]{13cm}{\footnotesize
      
      \vspace{\dp0}
      } \end{minipage} \\
      \\ \cdashline{2-3}

      & \begin{minipage}[t]{2cm}{Actual\\ Result}\end{minipage}   & 
      \begin{minipage}[t]{13cm}{\footnotesize
      
      \vspace{\dp0}
      } \end{minipage} \\
      \\ \cdashline{2-3}


      & Status          & Not Executed \\ \hline

    \end{longtable}


    \paragraph{Test Case LVV-T1436 - LDM-503-10a: Notebook Aspect tests for LSP with Authentication and TAP
milestone
 }\mbox{}\\

Open  \href{https://jira.lsstcorp.org/secure/Tests.jspa#/testCase/LVV-T1436}{\textit{ LVV-T1436 } }
test case in Jira.

    This test case verifies that the Notebook Aspect of the Science Platform
is accessible to authorized users through a login process, and that TAP
searches can be performed from Python code in the Notebook Aspect.


    \textbf{ Preconditions}:\\
    

    Execution status: {\bf Not Executed }

    Final comment:\\


    Detailed step results:

    \begin{longtable}{p{1cm}p{2cm}p{13cm}}
    \hline
    {Step} & \multicolumn{2}{c}{Description, Results and Status}\\ \hline
      1 & Description &

      \begin{minipage}[t]{13cm}{\footnotesize
      If~\href{https://jira.lsstcorp.org/secure/Tests.jspa\#/testCase/LVV-T1334}{LVV-T1334
(1.0)} has just been carried out, the tester will already be logged in
to the Portal Aspect.\\
Otherwise, use a Web browser to navigate to the landing page of the LSP
instance under test, and click through to the Portal Aspect link. ~This
should trigger a login process; the tester should log in. ~Non-NCSA
credentials should be used (or have been used) to log in to the Portal
Aspect.

      \vspace{\dp0}
      } \end{minipage} \\
      \\ \cdashline{2-3}


      & Expected Result &

      \begin{minipage}[t]{13cm}{\footnotesize
      The web browser should display a Portal Aspect page with the user's name
noted in the upper right hand corner.

      \vspace{\dp0}
      } \end{minipage} \\
      \\ \cdashline{2-3}

      & \begin{minipage}[t]{2cm}{Actual\\ Result}\end{minipage}   & 
      \begin{minipage}[t]{13cm}{\footnotesize
      
      \vspace{\dp0}
      } \end{minipage} \\
      \\ \cdashline{2-3}


      & Status          & Not Executed \\ \hline

      2 & Description &

      \begin{minipage}[t]{13cm}{\footnotesize
      Use the same Web browser (in a new page or tab) to navigate to the
landing page of the LSP instance under test, and click through to the
Notebook Aspect link. ~

      \vspace{\dp0}
      } \end{minipage} \\
      \\ \cdashline{2-3}


      & Expected Result &

      \begin{minipage}[t]{13cm}{\footnotesize
      No login credentials should be requested. ~A page allowing the creation
of a Notebook Aspect session should be visible.

      \vspace{\dp0}
      } \end{minipage} \\
      \\ \cdashline{2-3}

      & \begin{minipage}[t]{2cm}{Actual\\ Result}\end{minipage}   & 
      \begin{minipage}[t]{13cm}{\footnotesize
      
      \vspace{\dp0}
      } \end{minipage} \\
      \\ \cdashline{2-3}


      & Status          & Not Executed \\ \hline

      3 & Description &

      \begin{minipage}[t]{13cm}{\footnotesize
      Use the Notebook Aspect UI to create a ``small'' session using the most
recent ``recommended'' (weekly) release image.

      \vspace{\dp0}
      } \end{minipage} \\
      \\ \cdashline{2-3}


      & Expected Result &

      \begin{minipage}[t]{13cm}{\footnotesize
      The main JupyterLab UI should appear.

      \vspace{\dp0}
      } \end{minipage} \\
      \\ \cdashline{2-3}

      & \begin{minipage}[t]{2cm}{Actual\\ Result}\end{minipage}   & 
      \begin{minipage}[t]{13cm}{\footnotesize
      
      \vspace{\dp0}
      } \end{minipage} \\
      \\ \cdashline{2-3}


      & Status          & Not Executed \\ \hline

      4 & Description &

      \begin{minipage}[t]{13cm}{\footnotesize
      Close any Portal Aspect window/tab(s) that are open.

      \vspace{\dp0}
      } \end{minipage} \\
      \\ \cdashline{2-3}


      & Expected Result &

      \begin{minipage}[t]{13cm}{\footnotesize
      
      \vspace{\dp0}
      } \end{minipage} \\
      \\ \cdashline{2-3}

      & \begin{minipage}[t]{2cm}{Actual\\ Result}\end{minipage}   & 
      \begin{minipage}[t]{13cm}{\footnotesize
      
      \vspace{\dp0}
      } \end{minipage} \\
      \\ \cdashline{2-3}


      & Status          & Not Executed \\ \hline

      5 & Description &

      \begin{minipage}[t]{13cm}{\footnotesize
      Use the JupyterLab Terminal application to create a small file in the
user's home directory.

      \vspace{\dp0}
      } \end{minipage} \\
      \\ \cdashline{2-3}


      & Expected Result &

      \begin{minipage}[t]{13cm}{\footnotesize
      The test file should be visible in the JupyterLab file browser.

      \vspace{\dp0}
      } \end{minipage} \\
      \\ \cdashline{2-3}

      & \begin{minipage}[t]{2cm}{Actual\\ Result}\end{minipage}   & 
      \begin{minipage}[t]{13cm}{\footnotesize
      
      \vspace{\dp0}
      } \end{minipage} \\
      \\ \cdashline{2-3}


      & Status          & Not Executed \\ \hline

      6 & Description &

      \begin{minipage}[t]{13cm}{\footnotesize
      Log out of the Notebook Aspect. ~

      \vspace{\dp0}
      } \end{minipage} \\
      \\ \cdashline{2-3}


      & Expected Result &

      \begin{minipage}[t]{13cm}{\footnotesize
      
      \vspace{\dp0}
      } \end{minipage} \\
      \\ \cdashline{2-3}

      & \begin{minipage}[t]{2cm}{Actual\\ Result}\end{minipage}   & 
      \begin{minipage}[t]{13cm}{\footnotesize
      
      \vspace{\dp0}
      } \end{minipage} \\
      \\ \cdashline{2-3}


      & Status          & Not Executed \\ \hline

      7 & Description &

      \begin{minipage}[t]{13cm}{\footnotesize
      Navigate to the landing page for the LSP instance under test. ~Navigate
to the Portal Aspect from that page. ~(Do not log in if a login is
requested.)

      \vspace{\dp0}
      } \end{minipage} \\
      \\ \cdashline{2-3}


      & Expected Result &

      \begin{minipage}[t]{13cm}{\footnotesize
      A login should be requested when the Portal Aspect is accessed. ~(This
verifies that \emph{logout} is cross-Aspect.)

      \vspace{\dp0}
      } \end{minipage} \\
      \\ \cdashline{2-3}

      & \begin{minipage}[t]{2cm}{Actual\\ Result}\end{minipage}   & 
      \begin{minipage}[t]{13cm}{\footnotesize
      
      \vspace{\dp0}
      } \end{minipage} \\
      \\ \cdashline{2-3}


      & Status          & Not Executed \\ \hline

      8 & Description &

      \begin{minipage}[t]{13cm}{\footnotesize
      Close the login window and quit the web browser in use.

      \vspace{\dp0}
      } \end{minipage} \\
      \\ \cdashline{2-3}


      & Expected Result &

      \begin{minipage}[t]{13cm}{\footnotesize
      
      \vspace{\dp0}
      } \end{minipage} \\
      \\ \cdashline{2-3}

      & \begin{minipage}[t]{2cm}{Actual\\ Result}\end{minipage}   & 
      \begin{minipage}[t]{13cm}{\footnotesize
      
      \vspace{\dp0}
      } \end{minipage} \\
      \\ \cdashline{2-3}


      & Status          & Not Executed \\ \hline

      9 & Description &

      \begin{minipage}[t]{13cm}{\footnotesize
      Launch a web browser and navigate to the landing page for the LSP
instance under test. ~Navigate to the Notebook Aspect. ~When prompted
for a login, use NCSA credentials (for the same user as the non-NCSA
credentials used above). ~Request a session of the ``medium'' category
with the most recent ``recommended'' (weekly) release image.

      \vspace{\dp0}
      } \end{minipage} \\
      \\ \cdashline{2-3}


      & Expected Result &

      \begin{minipage}[t]{13cm}{\footnotesize
      The usual JupyterLab UI should be displayed.

      \vspace{\dp0}
      } \end{minipage} \\
      \\ \cdashline{2-3}

      & \begin{minipage}[t]{2cm}{Actual\\ Result}\end{minipage}   & 
      \begin{minipage}[t]{13cm}{\footnotesize
      
      \vspace{\dp0}
      } \end{minipage} \\
      \\ \cdashline{2-3}


      & Status          & Not Executed \\ \hline

      10 & Description &

      \begin{minipage}[t]{13cm}{\footnotesize
      Examine the JupyterLab file browser for the file created in \textbf{Step
5} above. ~If convenient (e.g., based on other distinctive files or
persistent settings), verify further that the same user environment has
been reached as with the non-NCSA credentials above.

      \vspace{\dp0}
      } \end{minipage} \\
      \\ \cdashline{2-3}


      & Expected Result &

      \begin{minipage}[t]{13cm}{\footnotesize
      The same file should be visible. ~(This verifies that the two sets of
credentials lead to the same Notebook Aspect user environment.)

      \vspace{\dp0}
      } \end{minipage} \\
      \\ \cdashline{2-3}

      & \begin{minipage}[t]{2cm}{Actual\\ Result}\end{minipage}   & 
      \begin{minipage}[t]{13cm}{\footnotesize
      
      \vspace{\dp0}
      } \end{minipage} \\
      \\ \cdashline{2-3}


      & Status          & Not Executed \\ \hline

      11 & Description &

      \begin{minipage}[t]{13cm}{\footnotesize
      Clone the TBD test notebook for LDM-503-10a into the user environment
from the TBD tag of the TBD Github repository. ~Note the SHA that
applies to the version of the test notebook that has been cloned.

      \vspace{\dp0}
      } \end{minipage} \\
      \\ \cdashline{2-3}


      & Expected Result &

      \begin{minipage}[t]{13cm}{\footnotesize
      
      \vspace{\dp0}
      } \end{minipage} \\
      \\ \cdashline{2-3}

      & \begin{minipage}[t]{2cm}{Actual\\ Result}\end{minipage}   & 
      \begin{minipage}[t]{13cm}{\footnotesize
      
      \vspace{\dp0}
      } \end{minipage} \\
      \\ \cdashline{2-3}


      & Status          & Not Executed \\ \hline

      12 & Description &

      \begin{minipage}[t]{13cm}{\footnotesize
      Open the test notebook and execute all of its steps.\\[2\baselineskip]

      \vspace{\dp0}
      } \end{minipage} \\
      \\ \cdashline{2-3}


      & Expected Result &

      \begin{minipage}[t]{13cm}{\footnotesize
      
      \vspace{\dp0}
      } \end{minipage} \\
      \\ \cdashline{2-3}

      & \begin{minipage}[t]{2cm}{Actual\\ Result}\end{minipage}   & 
      \begin{minipage}[t]{13cm}{\footnotesize
      
      \vspace{\dp0}
      } \end{minipage} \\
      \\ \cdashline{2-3}


      & Status          & Not Executed \\ \hline

      13 & Description &

      \begin{minipage}[t]{13cm}{\footnotesize
      Note the success and/or failure indications that appear in the output of
the notebook.

      \vspace{\dp0}
      } \end{minipage} \\
      \\ \cdashline{2-3}


      & Expected Result &

      \begin{minipage}[t]{13cm}{\footnotesize
      
      \vspace{\dp0}
      } \end{minipage} \\
      \\ \cdashline{2-3}

      & \begin{minipage}[t]{2cm}{Actual\\ Result}\end{minipage}   & 
      \begin{minipage}[t]{13cm}{\footnotesize
      
      \vspace{\dp0}
      } \end{minipage} \\
      \\ \cdashline{2-3}


      & Status          & Not Executed \\ \hline

      14 & Description &

      \begin{minipage}[t]{13cm}{\footnotesize
      Save and close the test notebook. ~Save the fully-executed notebook in
TBD location as a record of the test.

      \vspace{\dp0}
      } \end{minipage} \\
      \\ \cdashline{2-3}


      & Expected Result &

      \begin{minipage}[t]{13cm}{\footnotesize
      
      \vspace{\dp0}
      } \end{minipage} \\
      \\ \cdashline{2-3}

      & \begin{minipage}[t]{2cm}{Actual\\ Result}\end{minipage}   & 
      \begin{minipage}[t]{13cm}{\footnotesize
      
      \vspace{\dp0}
      } \end{minipage} \\
      \\ \cdashline{2-3}


      & Status          & Not Executed \\ \hline

      15 & Description &

      \begin{minipage}[t]{13cm}{\footnotesize
      Without logging out, open a new browser window or tab, and navigate to
the Portal Aspect of the LSP instance under test. ~Verify that the
Portal Aspect can be accessed without a further login. ~Verify that the
username displayed at the upper right is the same one as in \textbf{Step
1} above.

      \vspace{\dp0}
      } \end{minipage} \\
      \\ \cdashline{2-3}


      & Expected Result &

      \begin{minipage}[t]{13cm}{\footnotesize
      
      \vspace{\dp0}
      } \end{minipage} \\
      \\ \cdashline{2-3}

      & \begin{minipage}[t]{2cm}{Actual\\ Result}\end{minipage}   & 
      \begin{minipage}[t]{13cm}{\footnotesize
      
      \vspace{\dp0}
      } \end{minipage} \\
      \\ \cdashline{2-3}


      & Status          & Not Executed \\ \hline

      16 & Description &

      \begin{minipage}[t]{13cm}{\footnotesize
      Log out of the Notebook Aspect, close the Portal Aspect windows, and
quit the Web browser in use.

      \vspace{\dp0}
      } \end{minipage} \\
      \\ \cdashline{2-3}


      & Expected Result &

      \begin{minipage}[t]{13cm}{\footnotesize
      
      \vspace{\dp0}
      } \end{minipage} \\
      \\ \cdashline{2-3}

      & \begin{minipage}[t]{2cm}{Actual\\ Result}\end{minipage}   & 
      \begin{minipage}[t]{13cm}{\footnotesize
      
      \vspace{\dp0}
      } \end{minipage} \\
      \\ \cdashline{2-3}


      & Status          & Not Executed \\ \hline

    \end{longtable}


    \paragraph{Test Case LVV-T1334 - LDM-503-10a: Portal Aspect tests for LSP with Authentication and TAP
milestone
 }\mbox{}\\

Open  \href{https://jira.lsstcorp.org/secure/Tests.jspa#/testCase/LVV-T1334}{\textit{ LVV-T1334 } }
test case in Jira.

    This test case verifies that the Portal Aspect of the Science Platform
is accessible to authorized users through a login process, and that TAP
searches can be performed from the Portal Aspect UI.\\[2\baselineskip]In
so doing and in conjunction with the other LDM-503-10a test cases
collected under LVV-P48, it addresses all or part of the following
requirements:

\begin{itemize}
\tightlist
\item
  DMS-LSP-REQ-0002, DMS-LSP-REQ-0005, DMS-LSP-REQ-0006,
  DMS-LSP-REQ-0020, DMS-LSP-REQ-0022, DMS-LSP-REQ-0023, DMS-LSP-REQ-0024
\item
  DMS-PRTL-REQ-0001, DMS-PRTL-REQ-0015, DMS-PRTL-REQ-0016,
  DMS-PRTL-REQ-0017, DMS-PRTL-REQ-0020, DMS-PRTL-REQ-0023,
  DMS-PRTL-REQ-0026, DMS-PRTL-REQ-0049, primarily
\end{itemize}


    \textbf{ Preconditions}:\\
    

    Execution status: {\bf Not Executed }

    Final comment:\\


    Detailed step results:

    \begin{longtable}{p{1cm}p{2cm}p{13cm}}
    \hline
    {Step} & \multicolumn{2}{c}{Description, Results and Status}\\ \hline
      1 & Description &

      \begin{minipage}[t]{13cm}{\footnotesize
      Navigate to the \url{https://lsst-lsp-stable.ncsa.illinois.edu/}
endpoint of the LSP at the LDF. ~From the displayed page, navigate to
the Portal Aspect.

      \vspace{\dp0}
      } \end{minipage} \\
      \\ \cdashline{2-3}


      & Expected Result &

      \begin{minipage}[t]{13cm}{\footnotesize
      A login screen should be displayed.

      \vspace{\dp0}
      } \end{minipage} \\
      \\ \cdashline{2-3}

      & \begin{minipage}[t]{2cm}{Actual\\ Result}\end{minipage}   & 
      \begin{minipage}[t]{13cm}{\footnotesize
      
      \vspace{\dp0}
      } \end{minipage} \\
      \\ \cdashline{2-3}


      & Status          & Not Executed \\ \hline

      2 & Description &

      \begin{minipage}[t]{13cm}{\footnotesize
      Log in to the Portal Aspect with NCSA credentials. ~Verify that a Portal
TAP search screen comes up. ~Note the user name displayed in the upper
left of the Portal. ~Log out.

      \vspace{\dp0}
      } \end{minipage} \\
      \\ \cdashline{2-3}


      & Expected Result &

      \begin{minipage}[t]{13cm}{\footnotesize
      Following login, the Portal Aspect TAP search screen should be
displayed, or a clearly visible UI element allowing one-click access to
that screen. ~A user name corresponding to the credentials entered
should be displayed.

      \vspace{\dp0}
      } \end{minipage} \\
      \\ \cdashline{2-3}

      & \begin{minipage}[t]{2cm}{Actual\\ Result}\end{minipage}   & 
      \begin{minipage}[t]{13cm}{\footnotesize
      
      \vspace{\dp0}
      } \end{minipage} \\
      \\ \cdashline{2-3}


      & Status          & Not Executed \\ \hline

      3 & Description &

      \begin{minipage}[t]{13cm}{\footnotesize
      Log in to the Portal Aspect with alternate credentials that are
associated with the same identity. ~

      \vspace{\dp0}
      } \end{minipage} \\
      \\ \cdashline{2-3}


      & Expected Result &

      \begin{minipage}[t]{13cm}{\footnotesize
      The Portal application should come up just as in the previous step; the
user name displayed in the upper left of the Portal should be the same
as in the previous step.

      \vspace{\dp0}
      } \end{minipage} \\
      \\ \cdashline{2-3}

      & \begin{minipage}[t]{2cm}{Actual\\ Result}\end{minipage}   & 
      \begin{minipage}[t]{13cm}{\footnotesize
      
      \vspace{\dp0}
      } \end{minipage} \\
      \\ \cdashline{2-3}


      & Status          & Not Executed \\ \hline

      4 & Description &

      \begin{minipage}[t]{13cm}{\footnotesize
      Navigate to the TAP search screen, if necessary, and ensure that the
LSST TAP service associated with the chosen LSP instance is selected.

      \vspace{\dp0}
      } \end{minipage} \\
      \\ \cdashline{2-3}


      & Expected Result &

      \begin{minipage}[t]{13cm}{\footnotesize
      A TAP search screen should either already be displayed after the
previous step, or should be displayed after a one-click action from the
Portal's initial page. ~On the TAP screen, a UI element allowing the
choice of TAP service to user should be available, and an LSST TAP
service associated with the LSP instance under test should be
pre-selected as the default.

      \vspace{\dp0}
      } \end{minipage} \\
      \\ \cdashline{2-3}

      & \begin{minipage}[t]{2cm}{Actual\\ Result}\end{minipage}   & 
      \begin{minipage}[t]{13cm}{\footnotesize
      
      \vspace{\dp0}
      } \end{minipage} \\
      \\ \cdashline{2-3}


      & Status          & Not Executed \\ \hline

      5 & Description &

      \begin{minipage}[t]{13cm}{\footnotesize
      Verify that the same WISE and SDSS catalog tables that were explored in
DMTR-52 are now visible in the TAP service.

      \vspace{\dp0}
      } \end{minipage} \\
      \\ \cdashline{2-3}


      & Expected Result &

      \begin{minipage}[t]{13cm}{\footnotesize
      The SDSS Stripe 82 2013 processing's deep detection and forced
photometry catalogs, and the WISE mission's principal catalog, forced
photometry catalog, and single-epoch source catalog should be
accessible.

      \vspace{\dp0}
      } \end{minipage} \\
      \\ \cdashline{2-3}

      & \begin{minipage}[t]{2cm}{Actual\\ Result}\end{minipage}   & 
      \begin{minipage}[t]{13cm}{\footnotesize
      
      \vspace{\dp0}
      } \end{minipage} \\
      \\ \cdashline{2-3}


      & Status          & Not Executed \\ \hline

      6 & Description &

      \begin{minipage}[t]{13cm}{\footnotesize
      Perform a TAP search on the AllWISE source catalog around the equatorial
coordinates (2, 0) (degrees), with a 30 arcminute radius, using the
Portal UI to specify the query.

      \vspace{\dp0}
      } \end{minipage} \\
      \\ \cdashline{2-3}


      & Expected Result &

      \begin{minipage}[t]{13cm}{\footnotesize
      This query should return about 12,000 rows of data. ~It should be
displayed in a table, as an overlay on a context image, and as a
configurable 2D density plot.

      \vspace{\dp0}
      } \end{minipage} \\
      \\ \cdashline{2-3}

      & \begin{minipage}[t]{2cm}{Actual\\ Result}\end{minipage}   & 
      \begin{minipage}[t]{13cm}{\footnotesize
      
      \vspace{\dp0}
      } \end{minipage} \\
      \\ \cdashline{2-3}


      & Status          & Not Executed \\ \hline

    \end{longtable}


\newpage
\appendix
%Make sure lsst-texmf/bin/generateAcronyms.py is in your path
\section{Acronyms used in this document}\label{sec:acronyms}
\addtocounter{table}{-1}
\begin{longtable}{|p{0.145\textwidth}|p{0.8\textwidth}|}\hline
\textbf{Acronym} & \textbf{Description}  \\\hline

2D & Two-dimensional \\\hline
API & Application Programming Interface \\\hline
DM & Data Management \\\hline
DMS & Data Management Subsystem \\\hline
DMTN & DM Technical Note \\\hline
DMTR & DM Test Report \\\hline
Data Management & The LSST Subsystem responsible for the Data Management System (DMS), which will capture, store, catalog, and serve the LSST dataset to the scientific community and public. The DM team is responsible for the DMS architecture, applications, middleware, infrastructure, algorithms, and Observatory Network Design. DM is a distributed team working at LSST and partner institutions, with the DM Subsystem Manager located at LSST headquarters in Tucson. \\\hline
Data Management Subsystem & The subsystems within Data Management may contain a defined combination of hardware, a software stack, a set of running processes, and the people who manage them: they are a major component of the DM System operations. Examples include the 'Archive Operations Subsystem' and the 'Data Processing Subsystem'"." \\\hline
IVOA & International Virtual-Observatory Alliance \\\hline
LDF & LSST Data Facility \\\hline
LDM & LSST Data Management (Document Handle) \\\hline
LSE & LSST Systems Engineering (Document Handle) \\\hline
LSP & LSST Science Platform \\\hline
LSST & Large Synoptic Survey Telescope \\\hline
NCSA & National Center for Supercomputing Applications \\\hline
PDAC & Prototype Data Access Center \\\hline
SDSS & Sloan Digital Sky Survey \\\hline
Science Platform & A set of integrated web applications and services deployed at the LSST Data Access Centers (DACs) through which the scientific community will access, visualize, and perform next-to-the-data analysis of the LSST data products. \\\hline
Scope & The work needed to be accomplished in order to deliver the product, service, or result with the specified features and functions \\\hline
Stripe 82 & A 2.5° wide equatorial band of sky covering roughly 300 square degrees that was observed repeatedly in 5 passbands during the course of the SDSS, In part for calibration purposes. \\\hline
TAP & Table Access Protocol \\\hline
TBD & To Be Defined (Determined) \\\hline
UI & User Interface \\\hline
WISE & Wide-field Survey Explorer \\\hline
epoch & Sky coordinate reference frame, e.g., J2000. Alternatively refers to a single observation (usually photometric, can be multi-band) of a variable source. \\\hline
forced photometry & A measurement of the photometric properties of a source, or expected source, with one or more parameters held fixed. Most often this means fixing the location of the center of the brightness profile (which may be known or predicted in advance), and measuring other properties such as total brightness, shape, and orientation. Forced photometry will be done for all Objects in the Data Release Production. \\\hline
\end{longtable}


\end{document}
