\addtocounter{table}{-1}
\begin{longtable}{|p{0.145\textwidth}|p{0.8\textwidth}|}\hline
\textbf{Acronym} & \textbf{Description}  \\\hline

2D & Two-dimensional \\\hline
API & Application Programming Interface \\\hline
DM & Data Management \\\hline
DMS & Data Management Subsystem \\\hline
DMTN & DM Technical Note \\\hline
DMTR & DM Test Report \\\hline
Data Management & The LSST Subsystem responsible for the Data Management System (DMS), which will capture, store, catalog, and serve the LSST dataset to the scientific community and public. The DM team is responsible for the DMS architecture, applications, middleware, infrastructure, algorithms, and Observatory Network Design. DM is a distributed team working at LSST and partner institutions, with the DM Subsystem Manager located at LSST headquarters in Tucson. \\\hline
Data Management Subsystem & The subsystems within Data Management may contain a defined combination of hardware, a software stack, a set of running processes, and the people who manage them: they are a major component of the DM System operations. Examples include the 'Archive Operations Subsystem' and the 'Data Processing Subsystem'"." \\\hline
IVOA & International Virtual-Observatory Alliance \\\hline
LDF & LSST Data Facility \\\hline
LDM & LSST Data Management (Document Handle) \\\hline
LSE & LSST Systems Engineering (Document Handle) \\\hline
LSP & LSST Science Platform \\\hline
LSST & Large Synoptic Survey Telescope \\\hline
NCSA & National Center for Supercomputing Applications \\\hline
PDAC & Prototype Data Access Center \\\hline
SDSS & Sloan Digital Sky Survey \\\hline
Science Platform & A set of integrated web applications and services deployed at the LSST Data Access Centers (DACs) through which the scientific community will access, visualize, and perform next-to-the-data analysis of the LSST data products. \\\hline
Scope & The work needed to be accomplished in order to deliver the product, service, or result with the specified features and functions \\\hline
Stripe 82 & A 2.5° wide equatorial band of sky covering roughly 300 square degrees that was observed repeatedly in 5 passbands during the course of the SDSS, In part for calibration purposes. \\\hline
TAP & Table Access Protocol \\\hline
TBD & To Be Defined (Determined) \\\hline
UI & User Interface \\\hline
WISE & Wide-field Survey Explorer \\\hline
epoch & Sky coordinate reference frame, e.g., J2000. Alternatively refers to a single observation (usually photometric, can be multi-band) of a variable source. \\\hline
forced photometry & A measurement of the photometric properties of a source, or expected source, with one or more parameters held fixed. Most often this means fixing the location of the center of the brightness profile (which may be known or predicted in advance), and measuring other properties such as total brightness, shape, and orientation. Forced photometry will be done for all Objects in the Data Release Production. \\\hline
\end{longtable}
